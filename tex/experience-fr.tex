\begin{rubric}{Expérience professionnelle}

  \subrubric{Cursus personnel}

  \entry*[\colorbox{green1}{Depuis 2007}]%
  \textbf{Doctorat Informatique et Applications:}
  \textit{Construction top-down de pyramides irrégulières pour le traitement d'images histologiques}. 
  Doctorat réalisé sous la direction de Luc BRUN (PR ENSICAEN, Caen) et Guillaume DAMIAND
  (CR CNRS LIRIS, Villeurbanne)
  
  \entry*[\colorbox{green1}{03--09/2007}]%
  Stage de fin d'études chez \textbf{Atos Origin} (33)
  \begin{itemize}
  \item activité de soutien/maintenance applicatif pour France Télécom
  \item développement J2EE en équipe pour le compte de France Télécom
  \end{itemize}
  
  \entry*[\colorbox{green1}{06--08/2006}]%
  Stage ingénieur chez \textbf{Bertin Technologies} (64)
  \begin{itemize}
  \item modélisation de scénarios d'incendie (FDS/Smokeview)
  \item participation à des études de dangers (simulations PHAST)
  \end{itemize}
  
  \entry*[\colorbox{green1}{06--07/2005}]%
  Stage ouvrier chez \textbf{Dassault Aviation Biarritz} (64)
  \begin{itemize}
  \item mise à jour de l'étude d'impact environnemental du site
  \item réalisation du bilan décennal de gestion des déchets de l'entreprise
  \end{itemize}
  
  \entry*[\colorbox{green1}{Étés 1999--2002}]%
  \textbf{Travaux saisonniers} en tant qu'aide-moniteur de voile, FFV Sanguinet (40)

  \newpage
  \subrubric{Enseignement}
  
  \entry*[\colorbox{green1}{2010-2011}]%
  \textbf{Demi ATER ENSICAEN (96H TD)}
  \begin{flushleft}
    \begin{small}
      \begin{tabular}{l l r}
        \hline\noalign{\smallskip}
        Année (Option)   & Intitulé de la formation & Heures ($\sim$ TD) \\
        \noalign{\smallskip}\hline\noalign{\smallskip}
        
        1A (Chimie) & Initiation à la programmation &  28 (23.3) \\
        & \detail{C, Algorithmique} & \\
        
        1A (Info) & Outils de Dévelopement Logiciel & 22 (14.6) \\
        & \detail{C, Makefile, Shell} & \\
        
        1A (Info) & Systèmes d'exploitation et Réseaux & 27 (18) \\
        & \detail{Forks, Pipes, Threads, Réseau TCP/IP} & \\
        
        1A (Info) & Java et programmation objet & 24 (16) \\
        & \detail{Eclipse, Java SE} & \\
        
        1A (Info) & Projets : Analyse et Traitement audio & 8 (8) \\
        & \detail{FFTW, Analyse spectrale} & \\

        2A (Info) & Conception d'interfaces graphiques & 18 (12) \\
        & \detail{Bibliothèques QT} & \\

        2A (Info) & Compilation & 10 (6.6) \\
        & \detail{Analyse lexicale et grammaticale (Lex/Yacc)} & \\
        \hline
      \end{tabular}
    \end{small}
  \end{flushleft}

  \entry*[\colorbox{green1}{2010}]%
  \textbf{1\iere{} année ENSICAEN (24H TP)}. 
  Encadrement de projets : création d'un Live-CD
  Linux, autocomplétion dans un shell, cross-scripting (XSS)
  
  \entry*[\colorbox{green1}{2009}]% 
  \textbf{1\iere{} année CPE Lyon (24H TP)}. Langage C++ :
  programmation orientée objet, STL, QT
     
\end{rubric}
